\section{Rubin Science Platform (including Qserv) for DP0.2} \label{sec:rspqserv}

During Data Preview 0.1 (June 2021) we deployed the Rubin Science Platform and high-scalability database Qserv on our Interim Data Facility in Google Cloud to 300 early users (about 300 Data Preview delegates) as well as project staff.
The initial set of services included catalog query support through all three aspects (Portal, VO API, Notebook), as well as Butler (data access middleware) access to image data in the Notebook Aspect.
The system has remained in production ever since servicing users and giving us a chance to settle on core operational processes, such as the cadence of updates.
In fact servicing our early access community on the IDF was one of the major reasons for our adoption of the so called "hybrid model" for operations, where the US DAC consists of an on-premises data center at SLAC (USDF) backing a cloud-based RSP deployment on Google Cloud (Cloud DF).

\subsection{RSP}\label{sec:rsp}

For DP0.2, the goals were to extend the DP0.1 deployment in the following ways:

\begin{itemize}
    \item Ingest data produced with our own Science Pipelines data model, with appropriate data model annotations based on IVOA standards;
    \item Support image search from the RSP Portal Aspect, allowing discovery and visualization of all the main DPDD image data types;
    \item Release a variety of improvements in the Portal Aspect to image display, charting, and data-type handling, as well as substantial support for IVOA DataLink;
    \item Provide an IVOA ObsTAP image metadata service in the API Aspect, based on new spatial-search capabilities added to Qserv, supporting searches by image type, location, time, and wavelength / filter band, with extensions for LSST-specific attributes;
    \item Provide authenticated image-access services based on IVOA DataLink, exposing the same data available via the Butler in the Notebook Aspect, with flexible support for adding additional DataLink services later;
    \item Deploy our new image cutout service, based on the IVOA SODA standard, available through the API and Portal Aspects;
    \item Release an authenticated HiPS service backed by HiPS images and Multi-Order Coverage (MOC) data generated by new capabilities in the Science Pipelines;
    \item Deploy infrastructure improvements, such as notebook idle session culling; and
    \item Double the number of external users (to about 600 delegates).
\end{itemize}

All these goals were met.
We had a soft launch about a week prior to the deadline to existing users, and after the July 4th weekend the Community Engagement Team onboarded the new users. Both of these activities went smoothly with no notable user-reported errors.

\input{qserv}

