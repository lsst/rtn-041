\documentclass[OPS,authoryear,toc]{lsstdoc}
\input{meta}

% Package imports go here.

% Local commands go here.

%If you want glossaries
%\input{aglossary.tex}
%\makeglossaries

\title{Data Preview 0.2 and Operations rehearsal for DRP.}

% Optional subtitle
% \setDocSubtitle{A subtitle}

\author{%
William O'Mullane
}

\setDocRef{RTN-041}
\setDocUpstreamLocation{\url{https://github.com/lsst/rtn-041}}

\date{\vcsDate}

% Optional: name of the document's curator
% \setDocCurator{The Curator of this Document}

\setDocAbstract{%
DM delivered software to operations to perform  processing of the DESC DC2 data as well as enhancements to the portal and Qserv for interaction with the results.  The release of this was called Data Preview 0.2 and the production of the data products and publication of them were carried out in an operational manner. This provides valuable insights for operational data releases. 
}

% Change history defined here.
% Order: oldest first.
% Fields: VERSION, DATE, DESCRIPTION, OWNER NAME.
% See LPM-51 for version number policy.
\setDocChangeRecord{%
  \addtohist{1}{YYYY-MM-DD}{Unreleased.}{William O'Mullane}
}


\begin{document}

% Create the title page.
\maketitle
% Frequently for a technote we do not want a title page  uncomment this to remove the title page and changelog.
% use \mkshorttitle to remove the extra pages

% ADD CONTENT HERE
% You can also use the \input command to include several content files.

\appendix
% Include all the relevant bib files.
% https://lsst-texmf.lsst.io/lsstdoc.html#bibliographies
\section{References} \label{sec:bib}
\renewcommand{\refname}{} % Suppress default Bibliography section
\bibliography{local,lsst,lsst-dm,refs_ads,refs,books}

% Make sure lsst-texmf/bin/generateAcronyms.py is in your path
\section{Acronyms} \label{sec:acronyms}
\addtocounter{table}{-1}
\begin{longtable}{p{0.145\textwidth}p{0.8\textwidth}}\hline
\textbf{Acronym} & \textbf{Description}  \\\hline

CAM & CAMera \\\hline
DC2 & Data Challenge 2 (DESC) \\\hline
DESC & Dark Energy Science Collaboration \\\hline
DM & Data Management \\\hline
DMCCB & DM Change Control Board \\\hline
DMTR & DM Test Report \\\hline
DP0 & Data Preview 0 \\\hline
DPLT & DP Leadership Team \\\hline
DRP & Data Release Production \\\hline
FrDF & French Data Facility \\\hline
LSST & Legacy Survey of Space and Time (formerly Large Synoptic Survey Telescope) \\\hline
NCSA & National Center for Supercomputing Applications \\\hline
OPS & Operations \\\hline
RTN & Rubin Technical Note \\\hline
T/CAM & Technical/Control (or Cost) Account Manager \\\hline
UKDF & United Kingdom Data Facility \\\hline
USDF & United States Data Facility \\\hline
\end{longtable}

% If you want glossary uncomment below -- comment out the two lines above
%\printglossaries





\end{document}
