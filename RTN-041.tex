\documentclass[OPS,authoryear,toc]{lsstdoc}
\input{meta}

% Package imports go here.

% Local commands go here.

%If you want glossaries
%\input{aglossary.tex}
%\makeglossaries

\title{Data Preview 0.2 and Operations rehearsal for DRP.}

% Optional subtitle
% \setDocSubtitle{A subtitle}

\author{%
William~O'Mullane,
Yusra~Alsayyad,
Hsin-Fang~Chiang,
Frossie~Economou,
Fritz~Mueller,
Tim~Jenness
}

\setDocRef{RTN-041}
\setDocUpstreamLocation{\url{https://github.com/lsst/rtn-041}}

\date{\vcsDate}

% Optional: name of the document's curator
% \setDocCurator{The Curator of this Document}

\setDocAbstract{%
DM delivered software to operations to perform  processing of the DESC DC2 data as well as enhancements to the portal and Qserv for interaction with the results.  The release of this was called Data Preview 0.2 and the production of the data products and publication of them were carried out in an operational manner. This provides valuable insights for operational data releases.
}

% Change history defined here.
% Order: oldest first.
% Fields: VERSION, DATE, DESCRIPTION, OWNER NAME.
% See LPM-51 for version number policy.
\setDocChangeRecord{%
  \addtohist{1}{2022-08-02}{Unreleased.}{William O'Mullane}
}


\begin{document}

% Create the title page.
\maketitle
% Frequently for a technote we do not want a title page  uncomment this to remove the title page and changelog.
% use \mkshorttitle to remove the extra pages

\section{Introduction}\label{sec:intro}

Between December 2021 and May 2022 the DESC DC2 \citep{2021ApJS..253...31L} was reprocessed with Rubin Science pipelines V23 \citedsp{DMTR-351}.\footnote{\url{https://pipelines.lsst.io/v/v23_0_2/index.html}}
Between May and June the catalogs were ingested to Qserv, tutorials and documentation were updated and the Data Preview 0.2 data release was made on time at the end of June.
A number of procedures were developed and practiced to achieve this.
Planning for DP0.1 and DP0.2 are in \citeds{RTN-001}.
We shall discuss the process in the following sections:

\begin{itemize}
\item  Management and communication is discussed in \secref{sec:management}
\item An overview of the processing is given in \secref{sec:processing}
\item Quality assurance and feedback to processing is discussed in  \secref{sec:qa}
\item Community engagement, tutorials and documentation are discussed in \secref{sec:cet}
\end{itemize}


\section{Management and communication} \label{sec:management}

Here we cover the management structures in place for DP0.2 this includes the groups and meetings like the change control for the pipeline version.


\section{DP0.2 processign on Google} \label{sec:processing}
text here

\section{Data Product Quality Assurance} \label{sec:qa}
text here

\section{Science Platform, user front end for DP0.2 } \label{sec:frontend}

DP0.1 upgrade image services any new issues  but mainly deployment and control, issue handling

\input{qserv}

input{cet}

\appendix
% Include all the relevant bib files.
% https://lsst-texmf.lsst.io/lsstdoc.html#bibliographies
\section{References} \label{sec:bib}
\renewcommand{\refname}{} % Suppress default Bibliography section
\bibliography{local,lsst,lsst-dm,refs_ads,refs,books}

% Make sure lsst-texmf/bin/generateAcronyms.py is in your path
\section{Acronyms} \label{sec:acronyms}
\addtocounter{table}{-1}
\begin{longtable}{p{0.145\textwidth}p{0.8\textwidth}}\hline
\textbf{Acronym} & \textbf{Description}  \\\hline

BPS & Batch Production Service \\\hline
CAM & CAMera \\\hline
CERN & European Organization for Nuclear Research \\\hline
CET & Community Engagement Team \\\hline
CPU & Central Processing Unit \\\hline
CSV & Comma Separated Values \\\hline
DC2 & Data Challenge 2 (DESC) \\\hline
DESC & Dark Energy Science Collaboration \\\hline
DM & Data Management \\\hline
DMCCB & DM Change Control Board \\\hline
DMTN & DM Technical Note \\\hline
DMTR & DM Test Report \\\hline
DP0 & Data Preview 0 \\\hline
DPLT & DP Leadership Team \\\hline
DRP & Data Release Production \\\hline
FrDF & French Data Facility \\\hline
GAR & Google Archive Registry \\\hline
GB & Gigabyte \\\hline
IDF & Interim Data Facility \\\hline
IVOA & International Virtual-Observatory Alliance \\\hline
LDM & LSST Data Management (Document Handle) \\\hline
LSST & Legacy Survey of Space and Time (formerly Large Synoptic Survey Telescope) \\\hline
NCSA & National Center for Supercomputing Applications \\\hline
OPS & Operations \\\hline
POSIX & Portable Operating System Interface \\\hline
PanDA &  Production ANd Distributed Analysis system \\\hline
RSP & Rubin Science Platform \\\hline
RTN & Rubin Technical Note \\\hline
SQL & Structured Query Language \\\hline
T/CAM & Technical/Control (or Cost) Account Manager \\\hline
UKDF & United Kingdom Data Facility \\\hline
USDF & United States Data Facility \\\hline
YAML & Yet Another Markup Language \\\hline
\end{longtable}

% If you want glossary uncomment below -- comment out the two lines above
%\printglossaries





\end{document}
